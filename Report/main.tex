\documentclass[12pt,openany]{book} % Default font size and left-justified equations

\usepackage[top=3cm,bottom=3cm,left=3.2cm,right=3.2cm,headsep=10pt,letterpaper]{geometry} % Page margins

\usepackage{xcolor} % Required for specifying colors by name
\definecolor{ocre}{RGB}{52,177,201} % Define the orange color used for highlighting throughout the book

% Font Settings
\usepackage{avant} % Use the Avantgarde font for headings
%\usepackage{times} % Use the Times font for headings
\usepackage{mathptmx} % Use the Adobe Times Roman as the default text font together with math symbols from the Sym­bol, Chancery and Com­puter Modern fonts

\usepackage{microtype} % Slightly tweak font spacing for aesthetics
\usepackage[utf8]{inputenc} % Required for including letters with accents
\usepackage[T1]{fontenc} % Use 8-bit encoding that has 256 glyphs

\usepackage[scaled]{helvet}
\renewcommand\familydefault{\sfdefault} 
\usepackage[T1]{fontenc}

% Bibliography
\usepackage[style=alphabetic,sorting=nyt,sortcites=true,autopunct=true,babel=hyphen,hyperref=true,abbreviate=false,backref=true,backend=biber]{biblatex}
\addbibresource{bibliography.bib} % BibTeX bibliography file
\defbibheading{bibempty}{}

\input{structure} % Insert the commands.tex file which contains the majority of the structure behind the template
\newcommand{\Cliente}{RO}
\newcommand{\Producto}{CORRECCION SIPM}
\newcommand{\Keywords}{['PHOTODETECTOR', 'RADIOPHARMACEUTICAL', 'PHOTOPEAK', 'TEMPERATURE', 'SENSOR', 'SIPM']}
\newcommand{\Periodo}{HASTA EL 21 DE JULIO}

\begin{document}

%----------------------------------------------------------------------------------------
%	TITLE PAGE
%----------------------------------------------------------------------------------------

\begingroup
\thispagestyle{empty}
\AddToShipoutPicture*{\put(0,0){\includegraphics[scale=1.2]{cover1}}} % Image background
\vspace*{5cm}
\begin{flushleft}
\par\normalfont\fontsize{35}{35}\sffamily\selectfont
\textbf{Vigilancia}\\
\textbf{Tecnológica}\\
{\LARGE Reporte, Julio 2017}\par % Book title
\vspace*{1cm}
{\Huge Cliente: \Cliente}\par % Author name
\end{flushleft}
\endgroup

%----------------------------------------------------------------------------------------
%	COPYRIGHT PAGE
%----------------------------------------------------------------------------------------

\newpage
~\vfill
\thispagestyle{empty}

\noindent \textsc{Technology forecasting report}\\

\noindent \textsc{www.easypatents.cl}\\ % URL

\noindent Copyright \copyright 2017 , EasyPatents Technologies all rights reserved \\

\noindent \textit{\today} % Fecha en que se envia

%----------------------------------------------------------------------------------------
%	TABLE OF CONTENTS
%----------------------------------------------------------------------------------------

\chapterimage{head1.png} % Table of contents heading image

\pagestyle{empty} % No headers

\tableofcontents % Print the table of contents itself

%\cleardoublepage % Forces the first chapter to start on an odd page so it's on the right

\pagestyle{fancy} % Print headers again

%----------------------------------------------------------------------------------------
%	CHAPTER 1
%----------------------------------------------------------------------------------------
\chapterimage{head1.png} % Chapter heading image

\chapter{Introducción}
El presente Informe es una vigilancia de las solicitudes de patente y modelo de utilidad que se han presentado ante la WORLD INTELLECTUAL PROPERTY ORGANIZATION (WIPO) utilizando el TRATADO DE COOPERACIÓN EN MATERIA DE PATENTES (PCT) y con prioridad vigente para entrar a los siguientes países resaltados en azul. Lista completa al final del Informe con fecha de entrada en vigencia del Tratado.


%\chapterimage{head1.png} % Chapter heading image

\chapter{Metodología}
Una vez realizado el procedimiento PCT ante la WIPO, el solicitante tiene el derecho a la solicitud de exclusividad de comercialización, distribución, importación, fabricación y uso del producto o proceso en cada uno de los países suscritos al acuerdo en un plazo de 30 meses desde la fecha de solicitud PCT o prioridad.\\

A continuación se informan las solicitudes de patentes de invención y modelos de utilidad través del título, resumen, fecha de publicación, fecha de solicitud, inventores y solicitantes obtenidas en relación al producto \textsc{\Producto} asociados a las palabras clave \textsc{\Keywords} de \textsc{\Cliente} y que han sido publicados en el período correspondiente al plazo máximo de prioridad \textsc{\Periodo}.

\chapter{Resultados}
 A continuación se mostrarán los 10 resultados más 
    relevantes de la búsqueda, en caso de existir, el resto de los resultados pueden ser vistos en anexos.
 
 \vspace{1cm}1- \textbf{Número de publicación:} \href{https://worldwide.espacenet.com/publicationDetails/biblio?DB=EPODOC&II=0&ND=3&adjacent=true&locale=en_EP&FT=D&date=20160331&CC=WO&NR=2009042480A1&KC=A1#}{\textcolor{blue}{WO2009042480A1}}\\ 
\textbf{Titulo:} PERFORATOR CHARGE WITH A CASE CONTAINING A REACTIVE MATERIAL\\ 
 
\textbf{Fecha de publicación:} 02 de Jan de 2009\\ 
\textbf{Fecha para ingreso a fases nacionales:} 03 de Jul de 2011\\ 
\textbf{Abstract:} a perforator \colorbox{yellow}{\colorbox{yellow}{charge}} includes a case formed of a material blend that includes a reactive material that is activated during \colorbox{yellow}{\colorbox{yellow}{explosive}} detonation of the perforator \colorbox{yellow}{\colorbox{yellow}{charge}}. an \colorbox{yellow}{\colorbox{yellow}{explosive}} and a \colorbox{yellow}{liner} are contained in the case, with the \colorbox{yellow}{liner} to collapse in response to detonation of the \colorbox{yellow}{\colorbox{yellow}{explosive}}.\\ 
 

 \vspace{1cm}2- \textbf{Número de publicación:} \href{https://worldwide.espacenet.com/publicationDetails/biblio?DB=EPODOC&II=0&ND=3&adjacent=true&locale=en_EP&FT=D&date=20160331&CC=WO&NR=2017035337A1&KC=A1#}{\textcolor{blue}{WO2017035337A1}}\\ 
\textbf{Titulo:} EFP DETONATING CORD\\ 
 
\textbf{Fecha de publicación:} 02 de Jan de 2017\\ 
\textbf{Fecha para ingreso a fases nacionales:} 03 de Jul de 2019\\ 
\textbf{Abstract:} a perforating tool (100) includes an encapsulated \colorbox{yellow}{shaped} \colorbox{yellow}{\colorbox{yellow}{charge}} (10) that has a bulkhead (26) with a reduced wall thickness section (34), a plate (55) having a shallow recess, and a detonating cord (50) having an energetic core (52). the energetic core forms the plate into an \colorbox{yellow}{\colorbox{yellow}{explosive}}ly formed perforator when detonated. the plate is positioned to direct the \colorbox{yellow}{\colorbox{yellow}{explosive}}ly formed perforator into the reduced wall thickness section.\\ 
 

 \vspace{1cm}3- \textbf{Número de publicación:} \href{https://worldwide.espacenet.com/publicationDetails/biblio?DB=EPODOC&II=0&ND=3&adjacent=true&locale=en_EP&FT=D&date=20160331&CC=WO&NR=2016132916A1&KC=A1#}{\textcolor{blue}{WO2016132916A1}}\\ 
\textbf{Titulo:} RETICULATE MOLDED RESIN ARTICLE\\ 
 
\textbf{Fecha de publicación:} 25 de Jan de 2016\\ 
\textbf{Fecha para ingreso a fases nacionales:} 25 de Jul de 2018\\ 
\textbf{Abstract:} this reticulate molded resin article (10) is used in order to enclose and protect a \colorbox{yellow}{\colorbox{yellow}{hollow}} tube member of a vehicle or small boat, and, in the case of being in the normal state in which no \colorbox{yellow}{\colorbox{yellow}{\colorbox{yellow}{load}}} is acting on the reticulate molded resin article (10), the reticulate molded resin article (10) comprises multiple first linear resin parts (11) which extend parallel to one another, and multiple second linear resin parts (12) which extend parallel to one another in a direction intersecting the first linear resin parts (11). the first linear resin parts (11) and the second linear resin parts (12) are alternately bonded in bond areas (13) where said parts intersect with each other. in an intersection, the orthographic projection direction p is defined as the direction which passes through the axial center of both the first linear resin part (11) and the second linear resin part (12) and is orthogonal to both axial centers. observing the first linear resin part (11) and the second linear resin part (12) from the orthographic projection direction p, the second surface area, which is the area of the bond area (13) between the first linear resin parts (11) and the second linear resin parts (12), is less than the first surface area, which is the surface area of the overlap between the first linear resin part (11) and the second linear resin part (12). the first linear resin parts (11) and the second linear resin parts (12) are formed from a material that contains a thermosetting resin.\\ 
 

 \vspace{1cm}4- \textbf{Número de publicación:} \href{https://worldwide.espacenet.com/publicationDetails/biblio?DB=EPODOC&II=0&ND=3&adjacent=true&locale=en_EP&FT=D&date=20160331&CC=WO&NR=2017109668A1&KC=A1#}{\textcolor{blue}{WO2017109668A1}}\\ 
\textbf{Titulo:} MAGNETIC STIRRER FOR FURNACE CONTAINING MOLTEN METAL AND FURNACE PROVIDED WITH SUCH A STIRRER\\ 
 
\textbf{Fecha de publicación:} 29 de Jan de 2017\\ 
\textbf{Fecha para ingreso a fases nacionales:} 30 de Jul de 2019\\ 
\textbf{Abstract:} a magnetic device (11) for a furnace (1) containing molten metal such as aluminum or the like, said furnace comprising a \colorbox{yellow}{\colorbox{yellow}{hollow}} body (2) with lateral walls (3) and a bottom wall (4), said furnace (1) being covered with refractory material and having a portion (6) thereof normally present on the bottom wall or in one of such walls (3, 4) which is made of metal permeable to the magnetic fields and covered with refractory layer, the device (11) acting as magnetic stirrer for the metal placed in the furnace being present at such wall portion (6), said device (11) being movable frontally and along such portion (6) permeable to the magnetic fields and comprising rotating magnetic field generator means adapted to generate a magnetic flux within the metal placed in the body of the furnace (1), said generator means (13) also preferably translating in front of said wall portion (6) along predefined guides (19). such generator means are constituted by a plurality of permanent magnets (15) carried by a \colorbox{yellow}{\colorbox{yellow}{\colorbox{yellow}{load}}}-bearing member (13) rotating around an axis (m, 57) thereof during said rotation and/or translation. a furnace with such stirrer device is also described and claimed.\\ 
 

 \vspace{1cm}5- \textbf{Número de publicación:} \href{https://worldwide.espacenet.com/publicationDetails/biblio?DB=EPODOC&II=0&ND=3&adjacent=true&locale=en_EP&FT=D&date=20160331&CC=WO&NR=2016122297A1&KC=A1#}{\textcolor{blue}{WO2016122297A1}}\\ 
\textbf{Titulo:} FUEL PUMP HAVING MONOLITHIC STRUCTURE\\ 
 
\textbf{Fecha de publicación:} 04 de Jan de 2016\\ 
\textbf{Fecha para ingreso a fases nacionales:} 04 de Jul de 2018\\ 
\textbf{Abstract:} the invention relates to a fuel drive pump, having a monolithic structure, for turbines of aircraft, comprising a pump body, wherein all the components are assembled to establish and control the pressure and flow in the pump, including an inlet port for supplying fuel at a controlled pressure, fuel return ports, fuel filter, pump rotor, port having an outlet pressure sensor, port having an inlet pressure sensor, pump cover having sealing system to maintain the pressure \colorbox{yellow}{\colorbox{yellow}{\colorbox{yellow}{load}}}, close the assembly and maintain the inlet pressure, pump impeller to establish the hydraulic outlet power, intermediate cap of the pump to regulate the pressure in the pump, and rotation arrow of the pump, having a \colorbox{yellow}{\colorbox{yellow}{hollow}} shaft for starting the driving of the system.\\ 
 

 \vspace{1cm}6- \textbf{Número de publicación:} \href{https://worldwide.espacenet.com/publicationDetails/biblio?DB=EPODOC&II=0&ND=3&adjacent=true&locale=en_EP&FT=D&date=20160331&CC=WO&NR=2017120362A1&KC=A1#}{\textcolor{blue}{WO2017120362A1}}\\ 
\textbf{Titulo:} VALUE-BASED TV ADVERTISING AUDIENCE EXCHANGE\\ 
 
\textbf{Fecha de publicación:} 13 de Jan de 2017\\ 
\textbf{Fecha para ingreso a fases nacionales:} 14 de Jul de 2019\\ 
\textbf{Abstract:} systems, methods and computer-readable media for a decentralized \colorbox{yellow}{application} system that enables participating parties to automate the buying and selling of tv media units and/or aggregated tv and premium video audiences is described. the value-based tv/premium video media exchange \colorbox{yellow}{application} system allows the participants to interact with the system directly and/or automate transactions and execution between systems, while ensuring proper governance over each participants own rules and economics associated with the transactions, as well as individual campaign constraints and requirements. the decentralized \colorbox{yellow}{application} system significantly lowers current transaction and execution barriers, timing and costs, while providing a highly accountable and trusted system across all of the exchange participants.\\ 
 

 \vspace{1cm}7- \textbf{Número de publicación:} \href{https://worldwide.espacenet.com/publicationDetails/biblio?DB=EPODOC&II=0&ND=3&adjacent=true&locale=en_EP&FT=D&date=20160331&CC=WO&NR=2017120163A1&KC=A1#}{\textcolor{blue}{WO2017120163A1}}\\ 
\textbf{Titulo:} EARTH REMOVAL AND SAND MINING SYSTEM AND METHOD\\ 
 
\textbf{Fecha de publicación:} 13 de Jan de 2017\\ 
\textbf{Fecha para ingreso a fases nacionales:} 14 de Jul de 2019\\ 
\textbf{Abstract:} an earth removal and sand \colorbox{yellow}{mining} system is provided, comprising a modular floating platform having an elongated u-\colorbox{yellow}{shaped} open channel defining an operating area; a gantry positioned above the operating area, wherein the gantry includes a crane and trolley operable along a predefined path within the operating area; a hoist extending from the trolley; and a pump operatively suspended from the trolley, wherein the pump includes a slurry dis\colorbox{yellow}{\colorbox{yellow}{charge}} hose. the platform includes an adjustable drag line connected to the pump to provide maximum control over the pump position for efficient operation.\\ 
 

 \vspace{1cm}8- \textbf{Número de publicación:} \href{https://worldwide.espacenet.com/publicationDetails/biblio?DB=EPODOC&II=0&ND=3&adjacent=true&locale=en_EP&FT=D&date=20160331&CC=WO&NR=0248420A2&KC=#}{\textcolor{blue}{WO0248420A2}}\\ 
\textbf{Titulo:} METHOD FOR PRODUCING COMPONENTS WITH A HIGH LOAD CAPACITY FROM TIAL ALLOYS\\ 
 
\textbf{Fecha de publicación:} 20 de Jan de 2002\\ 
\textbf{Fecha para ingreso a fases nacionales:} 20 de Jul de 2004\\ 
\textbf{Abstract:} the invention relates to a method for producing components with a high \colorbox{yellow}{\colorbox{yellow}{\colorbox{yellow}{load}}} capacity from alpha + gamma tial alloys, especially for producing components for aircraft engines or stationary gas turbines. according to this method, enclosed tial blanks of globular structure are \colorbox{yellow}{preformed} by isothermal primary forming in the alpha + gamma - or alpha phase area. the preforms are then \colorbox{yellow}{shaped} out into components with a predeterminable contour by means of at least one isothermal secondary forming process, with dynamic recrystallization in the alpha + gamma - or alpha phase area. the microstructure is adjusted by solution annealing the components in the alpha phase area and then cooling them off rapidly.\\ 
 

 \vspace{1cm}9- \textbf{Número de publicación:} \href{https://worldwide.espacenet.com/publicationDetails/biblio?DB=EPODOC&II=0&ND=3&adjacent=true&locale=en_EP&FT=D&date=20160331&CC=WO&NR=2005002975A2&KC=A2#}{\textcolor{blue}{WO2005002975A2}}\\ 
\textbf{Titulo:} BOMB PROOF GARBAGE RECEPTACLE AND BAG INSERTION METHOD FOR A GARBAGE CAN\\ 
 
\textbf{Fecha de publicación:} 13 de Jan de 2005\\ 
\textbf{Fecha para ingreso a fases nacionales:} 14 de Jul de 2007\\ 
\textbf{Abstract:} a garbage can for controlling an \colorbox{yellow}{\colorbox{yellow}{explosive}} \colorbox{yellow}{\colorbox{yellow}{charge}} generating an \colorbox{yellow}{\colorbox{yellow}{explosive}} force having an outer shell having vertical walls and a bottom; an inner shell having vertical walls and a bottom; and a middle wall between the outer shell and inner shell and a first layer of compressible material between the outer shell and middle shell and a second layer of compression material between the middle shell and the inner shell. the first layer may be different from the second layer. there may be more than two middle walls and more than a first and second layer of compressible material. the material itself may different from one layer to the other. the inner wall may be \colorbox{yellow}{shaped} to direct the \colorbox{yellow}{\colorbox{yellow}{explosive}} force. there is a \colorbox{yellow}{liner} having a lesser weight. the inner wall comprises a peg extending into the interior of the can and the \colorbox{yellow}{liner} defines a slot for receiving the peg at a first location near the bottom and at a second location farther from the bottom so that the \colorbox{yellow}{liner} may be raised and lowered by changing the connection of the peg from the first location to the second location. the slot is defined by a reinforced layer or a hardened surface defining or attached to the slot. preferably, the slot defines a bottom which is curved to direct the peg from the first location to the second location. a bag extending around the perimeter of the top of the linear and within are area defined interior to the perimeter top and the perimeter top defines a lip and wherein the can top perimeter defines a lip so that the bottom of the perimeter top contacts the lip of the can top.\\ 
 

 \vspace{1cm}10- \textbf{Número de publicación:} \href{https://worldwide.espacenet.com/publicationDetails/biblio?DB=EPODOC&II=0&ND=3&adjacent=true&locale=en_EP&FT=D&date=20160331&CC=WO&NR=2004070311A2&KC=A2#}{\textcolor{blue}{WO2004070311A2}}\\ 
\textbf{Titulo:} DOUBLE EXPLOSIVELY-FORMED RING (DEFR) WARHEAD\\ 
 
\textbf{Fecha de publicación:} 19 de Jan de 2004\\ 
\textbf{Fecha para ingreso a fases nacionales:} 19 de Jul de 2006\\ 
\textbf{Abstract:} a warhead configuration for forming a hole through a wall of a target, the warhead configuration comprising a \colorbox{yellow}{\colorbox{yellow}{charge}} of \colorbox{yellow}{\colorbox{yellow}{explosive}} material and a \colorbox{yellow}{liner}. the \colorbox{yellow}{\colorbox{yellow}{charge}} has an axis and a front surface. the front surface includes two annular front surface portions, an inner and an outer annular portion, circumscribing the axis. each of the annular front surface portions is configured so as to exhibit a concave profile as viewed in a cross-section through the \colorbox{yellow}{\colorbox{yellow}{charge}} parallel to the axis. the \colorbox{yellow}{liner} includes a first \colorbox{yellow}{liner} disposed adjacent to the inner annular portion and a second \colorbox{yellow}{liner} disposed adjacent to the outer annular portion such that, when the \colorbox{yellow}{\colorbox{yellow}{charge}} is detonated, material from the first \colorbox{yellow}{liner} is formed into a first expanding \colorbox{yellow}{\colorbox{yellow}{explosive}}ly formed ring and material from the second \colorbox{yellow}{liner} is formed into a second \colorbox{yellow}{\colorbox{yellow}{explosive}}ly formed ring.\\ 
 

 \vspace{1cm}11- \textbf{Número de publicación:} \href{https://worldwide.espacenet.com/publicationDetails/biblio?DB=EPODOC&II=0&ND=3&adjacent=true&locale=en_EP&FT=D&date=20160331&CC=WO&NR=03091184A2&KC=#}{\textcolor{blue}{WO03091184A2}}\\ 
\textbf{Titulo:} TIN ALLOY SHEATHED EXPLOSIVE DEVICE\\ 
 
\textbf{Fecha de publicación:} 06 de Jan de 2003\\ 
\textbf{Fecha para ingreso a fases nacionales:} 06 de Jul de 2005\\ 
\textbf{Abstract:} a substantially lead-free tin alloy composition that may be used with sheathed \colorbox{yellow}{\colorbox{yellow}{explosive}} devices in a variety of linear \colorbox{yellow}{\colorbox{yellow}{explosive}} devices, such as ignition cords, mild detonating cords, and linear \colorbox{yellow}{shaped} \colorbox{yellow}{\colorbox{yellow}{charge}}s. the tin alloy may also be used as a \colorbox{yellow}{liner} for \colorbox{yellow}{shaped} \colorbox{yellow}{\colorbox{yellow}{explosive}} devices.\\ 
 


\end{document}
