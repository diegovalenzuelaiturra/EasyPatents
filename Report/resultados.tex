\chapter{Resultados}
 A continuación se mostrarán los 10 resultados más 
    relevantes de la búsqueda, en caso de existir, el resto de los resultados pueden ser vistos en anexos.
 
 \vspace{1cm}1- \textbf{Número de publicación:} \href{https://worldwide.espacenet.com/publicationDetails/biblio?DB=EPODOC&II=0&ND=3&adjacent=true&locale=en_EP&FT=D&date=20160331&CC=WO&NR=2016185699A1&KC=A1#}{\textcolor{blue}{WO2016185699A1}}\\ 
\textbf{Titulo:} INFRARED IMAGING DEVICE\\ 
 
\textbf{Fecha de publicación:} 24 de Jan de 2016\\ 
\textbf{Fecha para ingreso a fases nacionales:} 24 de Jul de 2018\\ 
\textbf{Abstract:} [problem] to correct, with a high degree of accuracy, for fluctuations in the values for pixel signals output by an infrared \colorbox{yellow}{sensor} (3) caused by \colorbox{yellow}{temperature} variations, without causing an increase in the number of components of an infrared imaging device (1). [solution] this infrared imaging device (1) comprises: an imaging optical system (2); an infrared \colorbox{yellow}{sensor} (3) wherein, by matching the center (31o) of the detection region (31) and the optical axis (o) of the imaging optical system, and making the length of the detection region (31) longer than the length of the imaging region (d) of the imaging optical system in one direction passing through the center (31o), an effective region (a) in which the detection region (31) and the imaging region (d) overlap, and reference regions (b) in which the detection region (31) and the imaging region (d) do not overlap, are provided within the detection region (31); and a signal correction unit (61) that corrects fluctuations in the values of pixel signals caused by \colorbox{yellow}{temperature} variations, by correcting values of pixel signals for effective pixels, which are pixels within the effective region (a), using the values for pixel signals for reference signals, which are pixels within the reference region (b).\\ 
 

 \vspace{1cm}2- \textbf{Número de publicación:} \href{https://worldwide.espacenet.com/publicationDetails/biblio?DB=EPODOC&II=0&ND=3&adjacent=true&locale=en_EP&FT=D&date=20160331&CC=WO&NR=2017039280A1&KC=A1#}{\textcolor{blue}{WO2017039280A1}}\\ 
\textbf{Titulo:} SAMPLE PRETREATMENT SYSTEM AND CONTROL METHOD THEREFOR\\ 
 
\textbf{Fecha de publicación:} 09 de Jan de 2017\\ 
\textbf{Fecha para ingreso a fases nacionales:} 10 de Jul de 2019\\ 
\textbf{Abstract:} a sample pretreatment system and a control method therefor are disclosed. the sample pretreatment system and control method therefor according to the present invention minimize errors which may occur when an operator personally and manually proceeds, can secure the accuracy and uniformity of pretreatment and test results of a sample, and can provide a user-friendly testing environment as well as facilitating the convenience of work by simply carrying out the mixing and discharge of the sample. in addition, the present invention can uniformly maintain and control the pressure within a chamber so as to prevent the sample from bursting out in case of a drastic pressure change in the chamber, increases heat transfer to the sample accommodated in the chamber, thereby heats to a desired \colorbox{yellow}{temperature} within a short time, and thus can increase the mixing and reaction efficiency of the sample. further, the present invention increases the mixing effect of the sample using a magnetic force, can minimize mechanical driving, and can discharge the sample in a fixed quantity after the pretreatment of the sample.\\ 
 

 \vspace{1cm}3- \textbf{Número de publicación:} \href{https://worldwide.espacenet.com/publicationDetails/biblio?DB=EPODOC&II=0&ND=3&adjacent=true&locale=en_EP&FT=D&date=20160331&CC=WO&NR=2017039279A1&KC=A1#}{\textcolor{blue}{WO2017039279A1}}\\ 
\textbf{Titulo:} SAMPLE PRETREATMENT MODULE AND SAMPLE PRETREATMENT METHOD USING SAME\\ 
 
\textbf{Fecha de publicación:} 09 de Jan de 2017\\ 
\textbf{Fecha para ingreso a fases nacionales:} 10 de Jul de 2019\\ 
\textbf{Abstract:} disclosed are a sample pretreatment module and a sample pretreatment method using same. the sample pretreatment module and the sample pretreatment method using same according to the present invention can minimize errors which can occur when a worker performs work manually, ensure accuracy and uniformity in sample pretreatment and test outcomes and, by enabling easy sample mixing and discharging, enhance working convenience and provide a user-friendly test environment. in addition, the module and the method can control and maintain a uniform pressure in the chamber so as to be able to prevent the outpouring of a sample even in the case of a drastic pressure change in the chamber, and can better transfer heat to the sample accommodated in the chamber, thereby heating the sample to a desired \colorbox{yellow}{temperature} within a short time and thus increasing the mixing and reaction efficiencies of the sample. also, the module and the method use a magnetic force, resulting in more effective sample mixing and minimal mechanical operation, and enable an exact amount of the sample to be released after the pretreatment thereof.\\ 
 

 \vspace{1cm}4- \textbf{Número de publicación:} \href{https://worldwide.espacenet.com/publicationDetails/biblio?DB=EPODOC&II=0&ND=3&adjacent=true&locale=en_EP&FT=D&date=20160331&CC=WO&NR=2017061273A1&KC=A1#}{\textcolor{blue}{WO2017061273A1}}\\ 
\textbf{Titulo:} IMAGING DEVICE, MANUFACTURING METHOD\\ 
 
\textbf{Fecha de publicación:} 13 de Jan de 2017\\ 
\textbf{Fecha para ingreso a fases nacionales:} 14 de Jul de 2019\\ 
\textbf{Abstract:} the present technology pertains to an imaging device and a manufacturing method, enabling improvement of sensitivity of an imaging device that utilizes infrared rays. the present invention is provided with: a light receiving element array having multiple light receiving elements which are arranged in an array and which are each formed of a compound semiconductor that has the light receiving sensitivity in the infrared range; a signal processing circuit that processes signals from the light receiving element; an upper electrode formed at the light receiving surface side of the light receiving elements; and a lower electrode that forms a pair with the upper electrode. the light receiving element array and the signal processing circuit are joined to each other via a film formed of a predetermined material. the upper electrode and the signal processing circuit are connected to each other via a through-hole partially penetrating the light receiving elements. the lower electrode is an electrode in common with the light receiving elements arranged in the light receiving element array. the present technology is applicable to an infrared ray \colorbox{yellow}{sensor}.\\ 
 

 \vspace{1cm}5- \textbf{Número de publicación:} \href{https://worldwide.espacenet.com/publicationDetails/biblio?DB=EPODOC&II=0&ND=3&adjacent=true&locale=en_EP&FT=D&date=20160331&CC=WO&NR=2017053786A1&KC=A1#}{\textcolor{blue}{WO2017053786A1}}\\ 
\textbf{Titulo:} TEMPERATURE CONTROL DEVICE AND PROCESS CONTROL APPARATUS INCLUDING A TEMPERATURE CONTROL DEVICE\\ 
 
\textbf{Fecha de publicación:} 30 de Jan de 2017\\ 
\textbf{Fecha para ingreso a fases nacionales:} 31 de Jul de 2019\\ 
\textbf{Abstract:} a process control apparatus includes a housing, a process control device disposed in the housing, and a \colorbox{yellow}{temperature} control device operably coupled to the housing for regulating a \colorbox{yellow}{temperature} of an atmosphere internal to the housing. the \colorbox{yellow}{temperature} control device includes a vortex tube and a flow control valve. the flow control valve is coupled to the vortex tube and includes a \colorbox{yellow}{temperature} sensing feature configured to sense a \colorbox{yellow}{temperature} of an atmosphere internal to the housing and configured to move a control element of the flow control valve based on the sensed \colorbox{yellow}{temperature} between a plurality of positions to selectively direct the flow of fluid from the first and second vortex outlets to the atmosphere internal to the housing.\\ 
 

 \vspace{1cm}6- \textbf{Número de publicación:} \href{https://worldwide.espacenet.com/publicationDetails/biblio?DB=EPODOC&II=0&ND=3&adjacent=true&locale=en_EP&FT=D&date=20160331&CC=WO&NR=2016190920A2&KC=A2#}{\textcolor{blue}{WO2016190920A2}}\\ 
\textbf{Titulo:} A SENSOR FOR MEASUREMENTS USING JOHNSON NOISE IN MATERIALS\\ 
 
\textbf{Fecha de publicación:} 01 de Jan de 2016\\ 
\textbf{Fecha para ingreso a fases nacionales:} 01 de Jul de 2018\\ 
\textbf{Abstract:} a method of making measurements includes providing a \colorbox{yellow}{sensor} with at least one solid state electronic spin; irradiating the \colorbox{yellow}{sensor} with radiation from an electromagnetic radiation source that manipulates the solid state electronic spins to produce spin-dependent fluorescence, wherein the spin-dependent fluorescence decays as a function of relaxation time; providing a target material in the proximity of the \colorbox{yellow}{sensor}, wherein, thermally induced currents (johnson noise) present in the target material alters the fluorescence decay of the solid state electronic spins as a function of relaxation time; and determining a difference in the solid state spins spin-dependent fluorescence decay in the presence and absence of the target material and correlating the difference with a property of the \colorbox{yellow}{sensor} and/or target material.\\ 
 

 \vspace{1cm}7- \textbf{Número de publicación:} \href{https://worldwide.espacenet.com/publicationDetails/biblio?DB=EPODOC&II=0&ND=3&adjacent=true&locale=en_EP&FT=D&date=20160331&CC=WO&NR=2016196524A1&KC=A1#}{\textcolor{blue}{WO2016196524A1}}\\ 
\textbf{Titulo:} FUNCTIONALIZED CRBON NANOTURE SENSORS, METHOD OF MAKING SAME AND USES THEREOF\\ 
 
\textbf{Fecha de publicación:} 08 de Jan de 2016\\ 
\textbf{Fecha para ingreso a fases nacionales:} 08 de Jul de 2018\\ 
\textbf{Abstract:} a carbon nanotube \colorbox{yellow}{sensor} device includes one or more carbon nanotubes and a functionalization layer. an outer surface of the one or more carbon nanotubes is coated with the functionalization layer and the functionalization layer includes a chemical compound that binds to one or more specific analytes. binding of the one or more specific analytes to the functionalization layer alters an electrical property of the carbon nanotube \colorbox{yellow}{sensor} device and contributes to their detection. the functionalization layer includes a first layer stacked onto an outer surface of the carbon nanotubes, a second layer stacked onto the first layer and a third layer stacked onto the second layer. the first layer enables stacking of a polymer onto the carbon nanotubes. the second layer includes the polymer and the third layer includes the chemical compound that binds to the one or more a specific analytes.\\ 
 

 \vspace{1cm}8- \textbf{Número de publicación:} \href{https://worldwide.espacenet.com/publicationDetails/biblio?DB=EPODOC&II=0&ND=3&adjacent=true&locale=en_EP&FT=D&date=20160331&CC=WO&NR=2017108230A1&KC=A1#}{\textcolor{blue}{WO2017108230A1}}\\ 
\textbf{Titulo:} METHOD AND SYSTEM FOR A CONTACTLESS TEMPERATURE MEASUREMENT\\ 
 
\textbf{Fecha de publicación:} 29 de Jan de 2017\\ 
\textbf{Fecha para ingreso a fases nacionales:} 30 de Jul de 2019\\ 
\textbf{Abstract:} the invention relates to an infrared measuring system (100) for ascertaining the \colorbox{yellow}{temperature} of an object (105) in a contactless manner, comprising an infrared \colorbox{yellow}{sensor} (115) for sensing an incident infrared radiation (110), a \colorbox{yellow}{temperature} \colorbox{yellow}{sensor} (125) for sensing a \colorbox{yellow}{temperature} of the infrared measuring system, and a processing device for carrying out a measurement method. the measuring method has the steps of determining the infrared radiation striking the infrared \colorbox{yellow}{sensor}; determining a \colorbox{yellow}{temperature} of the infrared measuring system; determining the \colorbox{yellow}{temperature} of the object on the basis of the determined radiation and the determined \colorbox{yellow}{temperature} of the infrared measuring system; and correcting the determined \colorbox{yellow}{temperature} of the infrared measuring system in a time-based manner in order to compensate for the influence of a \colorbox{yellow}{temperature} distribution in the infrared measuring system.\\ 
 

 \vspace{1cm}9- \textbf{Número de publicación:} \href{https://worldwide.espacenet.com/publicationDetails/biblio?DB=EPODOC&II=0&ND=3&adjacent=true&locale=en_EP&FT=D&date=20160331&CC=WO&NR=2017089020A1&KC=A1#}{\textcolor{blue}{WO2017089020A1}}\\ 
\textbf{Titulo:} DEVICE AND METHOD FOR ANALYSING AMBIENT AIR\\ 
 
\textbf{Fecha de publicación:} 01 de Jan de 2017\\ 
\textbf{Fecha para ingreso a fases nacionales:} 02 de Jul de 2019\\ 
\textbf{Abstract:} the invention relates to a situation-related analysis of the quality of ambient air. to this end, the activity of an air \colorbox{yellow}{sensor} for measuring a gas or a particle content in the ambient air can be adapted according to the situation based on at least one other environment related parameter.\\ 
 

 \vspace{1cm}10- \textbf{Número de publicación:} \href{https://worldwide.espacenet.com/publicationDetails/biblio?DB=EPODOC&II=0&ND=3&adjacent=true&locale=en_EP&FT=D&date=20160331&CC=WO&NR=2017046942A1&KC=A1#}{\textcolor{blue}{WO2017046942A1}}\\ 
\textbf{Titulo:} THERMAL BARRIER COATING AND POWER GENERATION SYSTEM\\ 
 
\textbf{Fecha de publicación:} 23 de Jan de 2017\\ 
\textbf{Fecha para ingreso a fases nacionales:} 24 de Jul de 2019\\ 
\textbf{Abstract:} [problem] to provide a thermal barrier coating which can be used on a member in a system used under high-\colorbox{yellow}{temperature}, high-pressure conditions, such as a supercritical co2 turbine, and is made of a highly durable ceramic. [solution] a thermal barrier coating according to an embodiment is characterized by an outermost ceramic layer having an uneven shape with an rz of not less than 0.1 mm. projections and recesses formed on the surface of the ceramic thermal barrier coating improve sealability and reduce hydrodynamic forces which make a rotor unstable. the projections and recesses can be formed by depositing a material for the ceramic layer by thermal spraying, etc., thereby simplify the process.\\ 
 

 \vspace{1cm}11- \textbf{Número de publicación:} \href{https://worldwide.espacenet.com/publicationDetails/biblio?DB=EPODOC&II=0&ND=3&adjacent=true&locale=en_EP&FT=D&date=20160331&CC=WO&NR=2017108256A1&KC=A1#}{\textcolor{blue}{WO2017108256A1}}\\ 
\textbf{Titulo:} SENSOR ELEMENT FOR DETECTING AT LEAST ONE PROPERTY OF A MEASURING GAS IN A MEASURING GAS CHAMBER\\ 
 
\textbf{Fecha de publicación:} 29 de Jan de 2017\\ 
\textbf{Fecha para ingreso a fases nacionales:} 30 de Jul de 2019\\ 
\textbf{Abstract:} the invention relates to a \colorbox{yellow}{sensor} element (10) for detecting at least one property of a measuring gas in a measuring gas chamber, in particular for detecting a proportion of the gas component in the measuring gas or a \colorbox{yellow}{temperature} of the measuring gas. the \colorbox{yellow}{sensor} element (10) comprises a ceramic layer structure (12) with at least one electrochemical cell, wherein the electrochemical cell has at least one first electrode (16), one second electrode (18), and at least one solid electrolyte (14) connecting the first electrode (16) and the second electrode (18). an electrode cavity (24) is formed in the layer structure (12). the second electrode (18) is arranged in the layer structure (12) such that the second electrode (18) faces the electrode cavity (24). the second electrode (18) has at least one first outer diameter (58) which is greater than a first outer diameter (60) of the electrode cavity (24).\\ 
 
